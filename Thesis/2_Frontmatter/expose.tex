	\chapter*{Exposé}
	\section*{Kurzbeschreibung des Themas(Problemstellung)}
    Bei der Automatisierung von Integrationstests ist ein entscheidender Schritt das Definieren und Sichern von Testfällen. Nicht selten werden als Lösung Datenbanken oder Einfache Dokumente gewählt, die manuell gepflegt werden. Diese Art der Sicherung von Testfällen erweist sich als fehleranfällig, besonders was die Aussagekraft der Menge der Testfälle angeht, da nur schwer überprüft werden kann, wie hoch beispielsweise die Abdeckungsrate ist. Zusätzlich ist das Erweitern und Warten solcher Dokumente sehr aufwändig. 
\section*{Zentrale Fragestellung der Arbeit}
\begin{itemize}
    \item Was sind Ontologien?
    \item Wo werden Ontologien verwendet?
    \item Wie kann man Ontologien einsetzen, um Testfälle zu definieren?
    \item Was sind die Grenzen von Ontologien?
    \item Beispiel-Testprojekt
    \begin{itemize}
        \item Welche Regeln gibt es?
        \item Welches Wissen soll in der Ontologie festgehalten werden?
        \item Wie kann man die Ontologie speichern/verwenden?
        \item Wie werden die Tests erweitert?
    \end{itemize}
\end{itemize}
\section*{Zielsetzung}
    Das Ziel der Arbeit ist es, Kenntnisse über Ontologien und ihre Verwendungszwecke zu bündeln und im Rahmen eines Beispiel-Testprojekts des BAYOOSOFT Access Managers einzusetzen, um Testfälle zu definieren. Daran kann man erkennen, wie effektiv diese Herangehensweise ist und welche Vor- und Nachteile gegenüber anderen Lösungen sich hieraus ergeben. 

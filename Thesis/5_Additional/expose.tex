%%
%%
%%
\chapter{Exposé}
\label{ch:expose}
%
\section{Kurzbeschreibung des Themas(Problemstellung)}
Umfangreiche Projekte wie zum Beispiel der BAYOOSOFT Access Manager bedürfen einer mindestens genauso umfangreichen Qualitätssicherung. Diese zu Automatisieren ist zunächst ein naheliegender Entschluss, wenn es darum geht Kosten und Zeit zu sparen und dabei trotzdem ein hohes maß an Qualität gewährleisten zu können. Oftmals wird jedoch der Aufwand und die Vorraussetzungen einer Testautomatisierung vernachlässigt, wenn die Entscheidung getroffen werden soll. Denn nicht alle Funktionen oder gar Projekte lassen sich automatisch testen, bzw. nicht alle manuellen Tests können effizient durch automatische ersetzt werden, ohne dass die Implementierung zu aufwändig wird.
\section{Zentrale Fragestellung der Arbeit}
\begin{itemize}
    \item Wie kann man testen?
    \item Was sind die Vorraussetzungen für automatisierte Tests?
    \item Was ist State-of-the-art?
    \item Was macht einen Test “nützlich”/”sinnvoll”?
    \item Ab wann sollte man über Testautomatisierung nachdenken (Größe des Projekts, Dynamik usw.)?
\end{itemize}
\section{Zielsetzung}
In der Arbeit soll erläutert werden, was Testautomatisierung für Entwickler und Projektleiter bedeutet und welcher Aufwand damit verbunden ist. Es soll geklärt werden, wann sich Testautomatisierung lohnt und welche Hürden ein solches Vorhaben blockieren können. Zusätzlich wird erklärt wie ein “guter” Test auszusehen hat und was beachtet werden muss, wenn man Integration, Unit und UI-Tests in einem ggf. großen Softwareprojekt anwenden muss.

\chapter{Einleitung}
\section{Problemstellung}
\section{Anlass}
\chapter{Möglichkeiten zu Testen}
\section{State-of-the-art}
\section{Unit-Tests, Integration-Tests und UI-Tests}
\chapter{Automatisierung}
\section{Vorgehen}
\subsection{TDD}
\section{Pipelines}
\chapter{Nutzen vs. Aufwand}
\section{Fallbeispiel Nutzen > Aufwand}
\section{Fallbeispiel Nutzen < Aufwand}
\section{Zusammenfassung}
\chapter{Fazit}
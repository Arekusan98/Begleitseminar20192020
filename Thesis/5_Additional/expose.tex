%%
%%
%%
\chapter{Exposé}
\label{ch:expose}
%
\section{Kurzbeschreibung des Themas(Problemstellung)}
Umfangreiche Projekte wie zum Beispiel der BAYOOSOFT Access Manager benötigen eine mindestens genauso umfangreiche Qualitätssicherung. Diese zu Automatisieren ist zunächst ein naheliegender Entschluss, wenn es darum geht, Kosten und Zeit zu sparen und dabei trotzdem ein hohes maß an Qualität gewährleisten zu können. Oftmals wird jedoch der Aufwand und die Vorraussetzungen einer Testautomatisierung unterschätzt, wenn die Entscheidung getroffen werden soll. Denn nicht alle Funktionen oder gar Projekte lassen sich automatisch testen, bzw. nicht alle manuellen Tests können effizient durch automatische ersetzt werden, ohne dass die Implementierung zu aufwändig wird.
\section{Zentrale Fragestellung der Arbeit}
\begin{itemize}
    \item Was sind Tests?
    \item Wie testet man?
    \item Wie automatisiert man Tests?
    \item Was sind die Vorraussetzungen für automatisierte Tests?
    \item Was ist State-of-the-art?
    \item Was macht einen Test “nützlich”/”sinnvoll”?
    \item Ab wann sollte man über Testautomatisierung nachdenken (Größe des Projekts, Dynamik usw.)?
\end{itemize}
\section{Zielsetzung}
    Ziel der Arbeit ist es, die Entscheidungsfindung bei der Frage nach einer Entwicklung von automatisierten Tests zu unterstützen. Dafür werden vor und Nachteile anhand verschiedener Beispiele abgewägt und gegenübergestellt. Dies lässt den Prozess erkennen, den man benötigt um eine fundierte Aussage treffen zu können, ob sich bei einem (ggf. großen) Projekt die Investition in automatisierte Tests lohnt. Zuvor werden noch verschiedene Grundlagen zum Thema Tests und Testautomatisierung dargelegt, was den aktuellen Stand der Technik, Methodiken, Richtlinien und Empfehlungen angeht, damit die Argumentation für und gegen Testautomatisierung fundiert innerhalb der Arbeit begründet und für den Leser nachvollziehbar gestaltet werden kann.

\chapter{Einleitung}
\section{Problemstellung}
\chapter{Grundlagen}
\section{Unit-Tests, Integration-Tests und UI-Tests}
\section{State-of-the-art}
\subsection{TDD}
\subsection{Continuous Integration}
\chapter{Diskussion}
\section{Fallbeispiel "Nutzen > Aufwand"}
\section{Fallbeispiel "Nutzen < Aufwand"}
\section{Fallbeispiel "Umsetzung nicht möglich"}
\section{Zusammenfassung}
\chapter{Fazit}
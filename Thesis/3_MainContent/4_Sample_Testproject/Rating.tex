\section{Bewertung}
Es gibt mehrere Dinge zu bewerten, wenn man sich die Ergebnisse der vorigen Kapitel anschaut. \\

Testfälle als Ontologie\newline
Zu Beginn wurde eine Ontologie erstellt, um einen Teil der Testfälle darzustellen. Dies konnte nach kurzer Zeit bewerkstelligt werden und ist nicht schwer umzusetzen gewesen. Beim "übersetzen" der bisherigen Testfälle können gegebenenfalls auch neue entdeckt werden, die man bisher nur noch nicht berücksichtigt hatte. Ebenso kann das Zusammenfassen gut aufgeteilt werden zwischen Teammitgliedern, da man die Ontologien am Ende mergen kann. Die Ontologien so klein und spezifisch zu lassen, wie es in dem Beispiel getan wurde, ist vermutlich nicht zu empfehlen, da sich die Stärken der Ontologie besonders darin zeigen, scheinbar voneinander unabhängiges Wissen strukturiert und losgelöst von einfachen Daten- und Attributtypen zu modellieren. Wäre für ein Produkt (in unserem Fall dem Access Manager) eine vollständigere Testonthologie vorhanden, könnte man damit vermutlich auch leichter die Testfälle integrieren.\\

Testfälle in einer Ontologie semantisch Suchen\newline
Die Testfälle innerhalb der Ontologie zu suchen, lässt sich (anhand der 4 Varianten der Suchunterstützung) theoretisch gut umsetzen. Die Testonthologien bieten alle Möglichkeiten, die Metadaten zur Verfügung zu stellen, die man benötigt um eine semantische Suche umzusetzen. Besonders dann, wenn die Testonthologie noch mehr Themenbereiche umfasst als in unserem Beispiel, da dann erst die explorative Suche und das Cross Referencing richtig zum tragen kommen.\\

Nutzen?\newline
Wie bereits bei der explorativen Suche angedeutet lässt der Nutzen nicht unbedingt sofort erkennen, doch gerade wenn es daran geht einen gescheiterten Testfall zu untersuchen, kann es wichtig sein, diesen im großen kontext nochmal zu betrachten. dabei kann eine ontologie mit semantischer suche von großer bedeutung sein. Ein User wird nicht korrekt berechtigt? Dann schaue ich mir den Test an und suche vllt genau nach "User1 lesen TestDomain" und erhalte alle Testergebnisse die auf diesen Suchbereich zutreffen. Ich kann auch die AD Tests suchen um zu schauen, ob andere Tests in diesem Bereich fehlgeschlagen sind. Weiterhin kann ich auch ähnliche Tests erhalten, die sehr ähnliches Verhalten zeigen aber nicht fehlgeschlagen sind, um zu ermitteln, was denn genau der Fehler war, der den Test um fehlschalgen gebracht hat. Und ebenfalls kann ich mich durch die Tests durchleiten lassen um zu sehen, wie sie aufgebaut sind. 

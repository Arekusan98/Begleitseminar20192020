\chapter{Bewertung}
Die vorigen Kapitel behandelten zwei große Themenbereiche mit Beispielen aus dem BAYOOSOFT Access Manager, in diesem Abschnitt sollen nun die Ergebnisse hinsichtlich ihren Nutzens und ihrem Aufwand bewertet werden. 
\section{Testfälle als Ontologie}
Zu Beginn wurde eine Ontologie erstellt, um eine kleine Anzahl Testfälle darzustellen. Der Aufwand des erstellens dieser kleinen Ontologie war gering, besonders da das Ermitteln der Testsfälle bereits geschehen war. Beim "Übersetzen" der bestehenden Testfälle können weitere Testfälle entdeckt werden, da das Umformen der Informationen zu neuen Einsichten in die Dynamik der Funktionen führt. Zusätzlich zu dem sowieso schon geringen Aufwand kann das Umformen der Testfälle in die Ontologie aufgeteilt werden, da sich Ontologien gut zusammenführen lassen. So kann man mehr Aspekte der Testsysteme berücksichtigen, da mehr Mitarbeiter ihre Ideen parallel einfließen lassen können. Die Ontologie aus dieser Hausarbeit ist absichtlich klein und kompakt gewählt worden, um mit wenig Umfang vor allem die Prinzipien und Konzepte von Ontologien und semantischer Suche zu demonstrieren. Eine Ontologien so klein und spezifisch zu modellieren ist jedoch nicht zu empfehlen, da sich die Stärken einer Ontologie besonders darin zeigen, scheinbar voneinander unabhängiges Wissen strukturiert und losgelöst von einfachen Daten- und Attributtypen zu zusammenzuführen. Außerdem ließ sich beim Entwickeln der Ontologie feststellen, dass desto mehr Daten in der Ontologie enthalten waren, desto leichter war es Punkte zu finden, an denen noch Informationen angefügt werden konnten um die Informationsammlung umfassender zu gestalten. \section{Testfälle in einer Ontologie semantisch Suchen}
Die aufgezeigten vier Varianten der Suchunterstützung durch semantische Suche mit Berücksichtigung von Metadaten aus der Testfallontologie lassen sich in der Theorie gut umsetzen. Alle Aspekte bezüglich Implikationen und Assoziation lassen sich bei den Testfällen genauso anwenden wie bei anderen Themengebieten, in denen semantische Suche eingesetzt wird. Desto größer die Testfallontologie ist, desto besser lässt sich die semantische Suche umsetzen. Denn Cross-Referencing und explorative Suche brauchen eine Vielzahl an Beziehungen und Klassen, um Assoziationen in einer nutzbar großen Menge herzustellen. 
\section{Nutzen}
Wie bereits bei der explorativen Suche angedeutet, lässt der Nutzen einer Testfallontologie und der semantischen Suche in diesem Bereich nicht unbedingt sofort erkennen. Semantische Suchen werden in der Regel eher im Webkontext eingesetzt, um beispielsweise Ergebnisse von Suchmaschinen zu optimieren \cite{Sack.2010}.\newline
Im Bereich des Produkttestens gibt es ebenfalls Einsatzmöglichkeiten. Zum Beispiel kann man bei gescheiterten Tests nach den betroffenen Testfällen suchen. Durch die semantische Suche kann zum einen garantiert werden, dass die Ergebnismenge sehr präzise nur die gewünschten Testfälle umfasst, zum anderen können aber auch ähnliche Tests berücksichtigt werden. Falls ein Test fehlschlägt, in dem das Active Directory verwendet wird, ist es unter Umständen von Nutzen, sich andere Active Directory Tests anzuschauen, da diese ähnlich vorgehen werden und so ermittelt werden kann, wo und warum ein Fehler aufgetreten ist. Aber auch die explorative Suche kann verwendet werden, um die Testfälle zugänglicher für Produktverantwortliche ohne technisches Know-How zu machen, da ohne Kenntnisse \glqq gestöbert\grqq{} werden kann. 

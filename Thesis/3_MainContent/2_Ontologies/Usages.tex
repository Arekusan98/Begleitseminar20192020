\section{Einsatzbereiche}
Aus den Eigenschaften von Ontologien ergeben sich mehrere Einsatzmöglichkeiten. Zum Einen werden Ontologien verwendet, um Wissen zu strukturieren und einen Datenaustausch zu ermöglichen. Außerdem sind sie nicht auf niedrige Datentypen angewiesen und ermöglichen so das Zusammenführen von umfangreichen heterogenen Daten. Sie erleichtern das Suchen innerhalb von Wissensbeständen und damit auch das Editieren einzelner Informationen. Ebenso können Begriffe miteinander verknüpft werden \cite{BernersLee.2001}, wodurch beispielsweise Synonyme oder Homonyme bei einer Suche berücksichtigt werden können.
%%%%%%%%%%%%%%%%YIKES
(quelle?)
%%%%%%%%%%%%%%%%YIKES
Einsatzbereiche von Ontologien sind zum Beispiel die Meeresbiologie. Forschungsergebnisse über Flora und Fauna werden gesammelt und unterstützen so die Forschung über Spezies und Biodiversität \cite{Garoufallou.2013}.
Ebenso werden im medizinischen Bereich Ontologien eingesetzt um Symptome, Medikamente und Patientendaten zu verknüpfen und dadurch Diagnosen zu ermitteln und Behandlungsmöglichkeiten zu suchen \cite{WolfgangHesse.2005}.\\

Ebenso sind Ontologien für das „Semantic Web“ von hoher Bedeutung, da sie das Verknüpfen verschiedener Seiten und Begriffe unterstützen \cite{BernersLee.2001} und durch das Verwalten von Metadaten das Anwenden einer semantischen Suche ermöglichen. \newline
In allen Bereichen, in denen Entscheidungen anhand von Erfahrung und Wissen getroffen werden oder Informationen strukturiert werden müssen, um sie effektiv nutzen zu können, rentiert sich der Einsatz von Ontologien.\newline
Das trifft auch auf den Bereich des Produkttestens zu. Das Wissen, dass hier verwaltet werden muss, ist die Vielzahl der Testfälle die sich für ein Produkt ermitteln lassen sowie der Zusammenhang zwischen den getesteten Funktionen mit den Modulen, aus denen ein System besteht. Eine Ontologie ist in der Lage, diese unterschiedlichen Informationen in einer Einheit unterzubringen und so das Arbeiten der Verantwortlichen für Produktqualität zu unterstützen.
\section{Einsatzbereiche}
Durch die eben aufgezeigten Eigenschaften von Ontologien ergeben sich verschiedene Möglichkeiten, diese einzusetzen. Zum einen werden Ontologien verwendet, um Wissen zu Strukturieren und Datenaustausch zu ermöglichen. Außerdem sind sie nicht auf niedrige Datentypen angewiesen und ermöglichen so das Zusammenführen von umfangreichen heterogenen Daten. Sie erleichtern das Suchen innerhalb von Wissensbeständen und damit auch das Editieren einzelner Informationen. Ebenso können Begriffe miteinander verknüpft werden \cite{BernersLee.2001}, wodurch beispielsweise Synonyme bei einer Suche berücksichtigen werden können, gleichzeitig aber auch homonyme ausgeschlossen werden können. (quelle?)
Eingesetzt werden Ontologien beispielsweise im Bereich der Meeresbiologie \cite{Garoufallou.2013} um Wissen über Meerestiere zu sammeln und die Forschung über diese Spezies und Biodiversitäten zu unterstützen. Ebenso werden im medizinischen Bereich Ontologien eingesetzt, um Symptome, Medikamente und Patientendaten zu verknüpfen und dadurch beispielsweise Diagnose und Therapiemöglichkeiten suchen zu können und das Personal zu unterstützen \cite{WolfgangHesse.2005}.\\

Ebenso sind Ontologien für das „Semantic Web“ von hoher Bedeutung, da sie die Verknüpfung zwischen verschiedenen Seiten und Begriffen ermöglichen \cite{BernersLee.2001} und semantische Suchen im Web so überhaupt erst umsetzbar sind. \newline
Alle Bereiche, in denen es gilt, Entscheidungen anhand von Erfahrung beziehungsweise Wissen zu treffen oder Wissen zu strukturieren, um damit Arbeiten zu können, können Ontologien einsetzen.\newline
Ebenso kann man also im Bereich des Produkttestens Ontologien einsetzen, um die Kenntnisse über die Vielzahl an Testfällen, die es bei einem größeren Produkt gibt, strukturiert zu bündeln und für semantische Suchen verwendbar zu machen. 

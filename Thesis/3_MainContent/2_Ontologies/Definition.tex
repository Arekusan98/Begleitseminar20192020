\chapter{Ontologien}
\section{Definition}
Ontologien definieren eine Reihe von repräsentativen Primitiven, mit denen ein Wissens- oder Diskursbereich modelliert werden kann. Sie bestehen aus Begriffen und Beziehungen, die einen bestimmten Problembereich darstellen. \cite{TomGruber.2009}\newline
Dadurch können Ontologien Kenntnisse und Wissen in einem Bereich auf eine strukturierte und nutzbare Art verfügbar machen. Die vielen Verknüpfungen innerhalb einer Ontologie verleihen ihr eine hohe semantische Ausdrucksstärke. Von vorhandenem Wissen können Rückschlüsse gezogen werden und weitere Beziehungen und Ableitungsregeln ermittelt werden. \cite{WolfgangHesse.2005}\newline
Ontologien lassen sich mit dreiteiligen Tupeln aufbauen, die nach dem Prinzip „Subjekt-Prädikat-Objekt“ formuliert sind. Hierbei entspricht das Subjekt einem Knoten, das Prädikat einer Kante und das Objekt entweder einem Literalwert oder einem weiteren Knoten. So wäre zum Beispiel „Stadt hat Einwohner“ ein Satz, der einer Ontologie im Bereich der Geografie entnommen werden kann, Wobei „Stadt“ und „Einwohner“ jeweils ein Knoten sind und „hat“ die Kante zwischen den Knoten darstellt. \cite{Wikipedia.31.10.201911:12}\newline
Um die Stuktur einer Ontologie nutzen zu können, muss sie mit Informationen gefüllt werden. Beispielsweise könnte bei der vorhin erwähnten Geografie Ontologie folgende Wissenssammlung vorliegen.\\

Kontinent <- ist in – Land <- ist in – Stadt <- wohnt in – Einwohner \\

Wird gefüllt mit \\

Europa <- Deutschland <- Frankfurt <- Max \\

Als Ausprägung der jeweiligen Begriffe, wodurch Rückschlüsse gezogen werden können wie z.B.: Max ist Europäer, auch wenn es keine direkte Verbindung zwischen Max und Europa gibt.
